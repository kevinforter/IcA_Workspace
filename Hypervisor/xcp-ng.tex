
%
% FH Technikum Wien
% !TEX encoding = UTF-8 Unicode
%
% Version V2.24 von 2024-12-19 otrebski
%
\documentclass[BMR,Seminar,ngerman,IEEE]{twbook}
\renewcommand*{\degreecourse}{FH Technikum Wien, Bachelorstudiengang BIF\\Lehrveranstaltung BIF-VZ-5-WS2025-INFC-EN}

% --- Encoding & fonts ---
\usepackage[utf8]{inputenc}
\usepackage[T1]{fontenc}

% --- Graphics, figures, urls, lists, tables, code ---
\usepackage{float}
\usepackage{placeins}
\usepackage{graphicx}
\usepackage{caption}
\usepackage{booktabs}
\usepackage{tabularx}
\usepackage{enumitem}
\usepackage{listings}
\usepackage[final]{microtype} % hilft gegen Overfull/Underfull
\emergencystretch=2em         % sanfter Puffer für Zeilenumbruch

% Screenshots hier hinein legen:
\graphicspath{{screenshots/}}

% --- Bibliography (twbook lädt biblatex via Hook, sobald inputenc geladen ist) ---
\addbibresource{Literatur.bib}

% --- Listings (inkl. YAML) ---
\lstdefinelanguage{yaml}{
  keywords={true,false,null,y,n},
  comment=[l]{\#},
  morestring=[b]',
  morestring=[b]",
  sensitive=false,
  showstringspaces=false,
  literate =    {---}{{\textcolor{gray}{---}}}3
                {>}{{\textcolor{gray}{>}}}1
                {|}{{\textcolor{gray}{|}}}1
                {:}{{\textcolor{blue}{:}}}1
                {-}{{\textcolor{gray}{-}}}1,
}
\lstset{
  basicstyle=\ttfamily\small,
  frame=single,
  breaklines=true,
  backgroundcolor=\color{gray!5},
  captionpos=b
}

% --- Convenience figure macros ---
\newcommand{\screenshotH}[3]{%
  \begin{figure}[H]
    \centering
    \includegraphics[width=\linewidth]{#1}%
    \caption{#2}%
    \label{fig:#3}%
  \end{figure}%
}
\newcommand{\placeholderfig}[2]{%
  \begin{figure}[htbp]
    \centering
    \fbox{\rule{0pt}{120pt}\rule{0.9\linewidth}{0pt}}%
    \caption{#1}%
    \label{fig:#2}%
  \end{figure}%
}

% Die nachfolgenden Pakete stellen sonst nicht benötigte Features zur Verfügung
\usepackage{blindtext}

% --- Hyperref am Schluss laden (robuster) ---
\usepackage{hyperref}

%
% Einträge für Deckblatt, Kurzfassung, etc.
%
\title{Hypervisor}
\author{David Veigel \and Kevin Forter \and Rahman Ridoy}
\studentnumber{XXXXXXXXXXXXXXX} % ggf. beide Nummern: 12345678, 87654321
\place{Wien}
% \acknowledgements{\blindtext}

\begin{document}

\maketitle

% ============================================================================
\onecolumn
\newpage

% ============================================================================
\chapter{Overview}

\section{XCP-ng}
XCP-ng steht für \textbf{Xen Cloud Platform – next generation}.  
Es basiert auf dem \textit{Citrix Hypervisor} (vormals XenServer) und ist eine vollständig \textbf{open-source Alternative}, entwickelt von \textbf{Vates}.  
XCP-ng wird für \textbf{Server-Virtualisierung}, \textbf{Private Clouds} und \textbf{Enterprise-Infrastrukturen} eingesetzt.  
Es ist vollständig kompatibel mit \textbf{Xen Orchestra (XO)} für webbasiertes Management.

% ----------------------------------------------------------------------------
\section{Type of Virtualization}
\textbf{Type:} Bare-metal / Type-1 Hypervisor  

\begin{itemize}[noitemsep]
  \item Wird direkt auf der Hardware installiert (nicht auf einem Host-Betriebssystem).
  \item Nutzt den Xen Project Hypervisor als Virtualisierungsschicht.
  \item Unterstützt \textbf{Full Virtualization} und \textbf{Paravirtualization} zur Performance-Optimierung.
\end{itemize}

% ----------------------------------------------------------------------------
\section{Core Functionalities}
\begin{itemize}[noitemsep]
  \item \textbf{VM Management:} Erstellen, Starten, Stoppen, Klonen und Löschen von VMs.
  \item \textbf{Snapshots:} Einfaches Erstellen und Wiederherstellen von Systemzuständen.
  \item \textbf{Live Migration:} Verschieben laufender VMs zwischen Hosts ohne Downtime.
  \item \textbf{Storage Management:} Unterstützung für SRs (Storage Repositories), LVM, NFS, iSCSI.
  \item \textbf{Network Virtualization:} Virtuelle Switches, VLANs und Bonding.
  \item \textbf{High Availability:} Automatisches Neustarten von VMs auf anderen Hosts bei Ausfall.
  \item \textbf{Dynamic Scalability:} CPU- und RAM-Zuweisungen im laufenden Betrieb anpassbar (Hotplug).
\end{itemize}

% ----------------------------------------------------------------------------
\section{Licensing Model}
\begin{itemize}[noitemsep]
  \item Open-Source und vollständig frei unter der \textbf{GPLv2}-Lizenz.
  \item Keine Funktionsunterschiede zwischen Community- und Enterprise-Version.
  \item Bezahlter Support über \textbf{Vates} für professionelle Umgebungen verfügbar.
\end{itemize}

% ----------------------------------------------------------------------------
\section{Server Virtualization, Support and Performance}
\begin{itemize}[noitemsep]
  \item Speziell für \textbf{Server-Virtualisierung} konzipiert.
  \item Unterstützt sowohl \textbf{Windows}- als auch \textbf{Linux}-Gastsysteme.
  \item Ressourcenlimits abhängig von der Hardware:
  \begin{itemize}
    \item Bis zu \textbf{288 CPUs} pro Host
    \item Bis zu \textbf{5\,TB RAM} pro Host
    \item Pro VM: typischerweise bis zu \textbf{32 vCPUs} und \textbf{128\,GB+ RAM} (konfigurierbar)
  \end{itemize}
  \item Clusterverwaltung über \textbf{Xen Orchestra} oder \textbf{XCP-ng Center}.
\end{itemize}

% ----------------------------------------------------------------------------
\section{Automation \& Monitoring}
\textbf{Automation:}
\begin{itemize}[noitemsep]
  \item Verwaltung über \textbf{Xen Orchestra API}, CLI-Tools oder Automatisierungsplattformen wie \textbf{Ansible} oder \textbf{Terraform}.
  \item Unterstützung für Skripting über die \texttt{xe}-Kommandozeile.
\end{itemize}

\textbf{Monitoring:}
\begin{itemize}[noitemsep]
  \item Integration mit \textbf{Xen Orchestra} für Dashboards und Leistungsmetriken.
  \item Kompatibel mit \textbf{Prometheus}, \textbf{Grafana} und \textbf{Zabbix} für erweitertes Monitoring.
  \item Echtzeitüberwachung von CPU-, RAM-, Netzwerk- und Datenträgernutzung pro VM und Host.
\end{itemize}
% ============================================================================

\end{document}